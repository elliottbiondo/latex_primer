\documentclass[12pt]{beamer}
 
\usetheme[white,sections]{Wisconsin}
\usepackage{fancyvrb}
\usepackage{color}
\usepackage[normalem]{ulem}
\usepackage{xcolor}
\usepackage{anyfontsize}
\usepackage{wrapfig}
\usepackage{graphicx}
\usepackage{verbatim}
\usepackage{framed}
\usepackage{listings}
\usepackage{showexpl}
\usepackage{xcolor}
\usepackage{lipsum}

\lstset{%
basicstyle=\scriptsize\ttfamily, numbersep=2mm, numbers=left, numberstyle=\tiny, % number style
breaklines=true,frame=single,framexleftmargin=0mm, xleftmargin=3mm,
prebreak = \raisebox{0ex}[0ex][0ex]{\ensuremath{\hookleftarrow}},
backgroundcolor=\color{green!5},frameround=fttt,escapeinside=??,
rulecolor=\color{red},
morekeywords={% Give key words here                                         % keywords
    maketitle},
keywordstyle=\color[rgb]{0,0,1},                    % keywords
        commentstyle=\color[rgb]{0.133,0.545,0.133},    % comments
        stringstyle=\color[rgb]{0.627,0.126,0.941}  % strings
%columns=fullflexible 
}%

\lstdefinelanguage{myTeX}{
  language=TeX,
  morekeywords={begin, frac, end, ref, item, label, hline, cite, author, title, year, bibliographystyle, bibliography, column, documentclass, usetheme, @book, lstset},
  morecomment=[l]{\%},
  sensitive=true
}

\usepackage{tikz}
\usetikzlibrary{shapes.arrows}
\tikzset{
    myarrow/.style={
        draw,
        fill=orange,
        single arrow,
        minimum height=4ex,
        single arrow head extend=1ex
    }
}
\tikzset{
    mydoublearrow/.style={
        draw,
        fill=orange,
        double arrow,
        minimum height=10.5ex,
        single arrow head extend=1ex
    }
}

\newcommand{\arrowup}{%
\tikz [baseline=-0.5ex]{\node [myarrow,rotate=90] {};}
}
\newcommand{\arrowdown}{%
\tikz [baseline=-1ex]{\node [myarrow,rotate=-90] {};}
}
\newcommand{\arrowright}{%
\tikz [baseline=-0.5ex]{\node [myarrow,rotate=0] {};}
}
\newcommand{\doublearrow}{%
\tikz [baseline=-1ex]{\node [mydoublearrow,rotate=0] {};}
}

\begin{document}
\newcommand*{\alphabet}{ABCDEFGHIJKLMNOPQRSTUVWXYZabcdefghijklmnopqrstuvwxyz}
\newlength{\highlightheight}
\newlength{\highlightdepth}
\newlength{\highlightmargin}
\setlength{\highlightmargin}{2pt}
\settoheight{\highlightheight}{\alphabet}
\settodepth{\highlightdepth}{\alphabet}
\addtolength{\highlightheight}{\highlightmargin}
\addtolength{\highlightdepth}{\highlightmargin}
\addtolength{\highlightheight}{\highlightdepth}
\setbeamertemplate{bibliography entry title}{}
\setbeamertemplate{bibliography entry location}{}
\setbeamertemplate{bibliography entry note}{}
\newcommand*{\Highlight}{\rlap{\textcolor{HighlightBackground}{\rule[-\highlightdepth]{\linewidth}{\highlightheight}}}}
\setbeamertemplate{bibliography item}[text]
\setbeamercolor{section in toc}{fg=white}
\setbeamercolor{bibliography entry author}{fg=black}
\setbeamercolor{bibliography item}{fg=black}
\setbeamercolor*{bibliography entry title}{fg=black}
%\setbeamerfont{section number projected}{size=\tiny}
\setbeamerfont{section number projected}{size=\fontsize{6}{6}\selectfont}
%\setbeamerfont{toc}{color=white}
\setbeamercolor{section number projected}{bg=UWRed,fg=white}
\hypersetup{linkcolor=white,urlcolor=white}

\AtBeginSection{\frame{\sectionpage}}
\defbeamertemplate{section page}{mine}[1][]{%
  \begin{centering}
    {\usebeamerfont{section name}\usebeamercolor[fg]{section name}#1}
    \vskip1em\par
    \begin{beamercolorbox}[sep=12pt,center]{part title}
      \usebeamerfont{section title}\insertsection\par
    \end{beamercolorbox}
  \end{centering}
}
\setbeamertemplate{section page}[mine]

%TITLE PAGE
%%%%%%%%%%%%%%%%%%%%%%%%%%%%%%%%%%%%%%%%%%%%%%%%%%%%%%%%%%%%%%%%%%%%%%%%%%%%%%%%
\title{\LaTeX \, and Beamer}   
\author{Elliott Biondo}
\institute{University of Wisconsin - Madison}
\date{March 27, 2015}
\frame[plain]{\titlepage \addtocounter{framenumber}{-1}} 
%%%%%%%%%%%%%%%%%%%%%%%%%%%%%%%%%%%%%%%%%%%%%%%%%%%%%%%%%%%%%%%%%%%%%%%%%%%%%%%%
\section{Introduction}
%%%%%%%%%%%%%%%%%%%%%%%%%%%%%%%%%%%%%%%%%%%%%%%%%%%%%%%%%%%%%%%%%%%%%%%%%%%%%%%%
\begin{frame}{\LaTeX and Beamer}
\LaTeX is a markup langauge used to create documents
Beamer is a \LaTeX class used to make presentations.
\begin{itemize}
\item Every aspect of this presentation was created in \LaTeX/Beamer.
\end{itemize}
\end{frame}

\begin{frame}{Why I use these tools}

\begin{itemize}
\item{Exacting control over every aspect of the document.}
\item{Powerful math mode}
\item{Automatic bibliography tool (Bib\TeX)}
\item{Automatic handling of numbering and cross references to figures, tables, equations, citations.}
\end{itemize}

\pause
``But I can do all of this in Microsoft Word $\dots$''

\end{frame}

\begin{frame}{Why I use these tools}

\end{frame}

%%%%%%%%%%%%%%%%%%%%%%%%%%%%%%%%%%%%%%%%%%%%%%%%%%%%%%%%%%%%%%%%%%%%%%%%%%%%%%%%
\section{\LaTeX}
%%%%%%%%%%%%%%%%%%%%%%%%%%%%%%%%%%%%%%%%%%%%%%%%%%%%%%%%%%%%%%%%%%%%%%%%%%%%%%%%
\begin{frame}[fragile]
\frametitle{Hello World!}

\begin{columns}[T]
\column{0.5\textwidth}
hello\_world.tex:
\lstinputlisting[language=myTeX, frame=single]{examples/hello_world.tex}
\column{0.05\textwidth}
\vspace{3cm}
\arrowright
\column{0.45\textwidth}
\fbox{\includegraphics[width=0.9\textwidth]{examples/hello_world.pdf}}
\vspace{0.5cm}
\end{columns}
\end{frame}
%%%%%%%%%%%%%%%%%%%%%%%%%%%%%%%%%%%%%%%%%%%%%%%%%%%%%%%%%%%%%%%%%%%%%%%%%%%%%%%%
\begin{frame}[fragile]
\frametitle{Let's make the font a little bigger...}

\begin{columns}[T]
\column{0.5\textwidth}
hello\_world2.tex:
\lstinputlisting[language=myTeX, frame=single]{examples/hello_world2.tex}
\column{0.05\textwidth}
\vspace{3cm}
\arrowright
\column{0.45\textwidth}
\fbox{\includegraphics[width=0.9\textwidth]{examples/hello_world2.pdf}}
%\fbox{\includegraphics[trim={5cm 5cm 0cm 3cm},clip]{examples/hello_world2.pdf}}
\vspace{0.5cm}
\end{columns}
\end{frame}
%%%%%%%%%%%%%%%%%%%%%%%%%%%%%%%%%%%%%%%%%%%%%%%%%%%%%%%%%%%%%%%%%%%%%%%%%%%%%%%%
\begin{frame}[fragile]
\frametitle{Some Basics}

\begin{columns}[T]
\column{0.52\textwidth}
lists.tex:
\lstinputlisting[language=myTeX, frame=single, firstline=6, lastline=12, firstnumber=6]{examples/lists.tex}
\column{0.05\textwidth}
\vspace{3cm}
\arrowright
\column{0.44\textwidth}
\fbox{\includegraphics[width=0.9\textwidth]{examples/lists.pdf}}
\vspace{0.5cm}
\end{columns}
\end{frame}
%%%%%%%%%%%%%%%%%%%%%%%%%%%%%%%%%%%%%%%%%%%%%%%%%%%%%%%%%%%%%%%%%%%%%%%%%%%%%%%%
\begin{frame}[fragile]
\frametitle{Some Basics}

\begin{columns}[T]
\column{0.5\textwidth}
lists2.tex:
\lstinputlisting[language=myTeX, frame=single, firstline=6, lastline=16, firstnumber=6]{examples/lists2.tex}
\column{0.05\textwidth}
\vspace{3cm}
\arrowright
\column{0.45\textwidth}
\fbox{\includegraphics[width=0.9\textwidth]{examples/lists2.pdf}}
\vspace{0.5cm}
\end{columns}
\end{frame}
%%%%%%%%%%%%%%%%%%%%%%%%%%%%%%%%%%%%%%%%%%%%%%%%%%%%%%%%%%%%%%%%%%%%%%%%%%%%%%%%
\begin{frame}[fragile]
\frametitle{Math}

\begin{columns}[T]
\column{0.5\textwidth}
math.tex:
\lstinputlisting[language=myTeX, frame=single, firstline=6, lastline=13, firstnumber=6]{examples/math.tex}
Note that \texttt{\textbackslash label} works with equations, tables, figures, code listings, etc.
\column{0.05\textwidth}
\vspace{3cm}
\arrowright
\column{0.45\textwidth}
\fbox{\includegraphics[width=0.9\textwidth]{examples/math.pdf}}
\vspace{0.5cm}
\end{columns}
\end{frame}

%%%%%%%%%%%%%%%%%%%%%%%%%%%%%%%%%%%%%%%%%%%%%%%%%%%%%%%%%%%%%%%%%%%%%%%%%%%%%%%%
\begin{frame}[fragile]
\frametitle{Figures}

\begin{columns}[T]
\column{0.5\textwidth}
figure.tex:
\lstinputlisting[language=myTeX, frame=single, firstline=1, lastline=20, firstnumber=1]{examples/figure.tex}
\column{0.05\textwidth}
\vspace{3cm}
\arrowright
\column{0.45\textwidth}
\fbox{\includegraphics[width=0.9\textwidth]{examples/figure.pdf}}
\vspace{0.5cm}
\end{columns}
\end{frame}

%%%%%%%%%%%%%%%%%%%%%%%%%%%%%%%%%%%%%%%%%%%%%%%%%%%%%%%%%%%%%%%%%%%%%%%%%%%%%%%%
\begin{frame}[fragile]
\frametitle{Tables}

\begin{columns}[T]
\column{0.5\textwidth}
table.tex:
\lstinputlisting[language=myTeX, frame=single, firstline=7, lastline=14, firstnumber=7]{examples/table.tex}
\column{0.05\textwidth}
\vspace{3cm}
\arrowright
\column{0.45\textwidth}
\fbox{\includegraphics[width=0.9\textwidth]{examples/table.pdf}}
\vspace{0.5cm}
\end{columns}
\end{frame}
%%%%%%%%%%%%%%%%%%%%%%%%%%%%%%%%%%%%%%%%%%%%%%%%%%%%%%%%%%%%%%%%%%%%%%%%%%%%%%%%
\begin{frame}[fragile]
\frametitle{Code listings}

\begin{columns}[T]
\column{0.5\textwidth}
listings.tex:
\lstinputlisting[language=myTeX, frame=single, firstline=9, lastline=15, firstnumber=9]{examples/listing.tex}
(with \texttt{\textbackslash usepackage\{listings\}}  in the preamble)
\column{0.05\textwidth}
\vspace{3cm}
\arrowright
\column{0.45\textwidth}
\fbox{\includegraphics[width=0.9\textwidth]{examples/listing.pdf}}
\vspace{0.5cm}
\end{columns}
\end{frame}

%%%%%%%%%%%%%%%%%%%%%%%%%%%%%%%%%%%%%%%%%%%%%%%%%%%%%%%%%%%%%%%%%%%%%%%%%%%%%%%%
\begin{frame}[fragile]
\frametitle{References}

\begin{columns}[T]
\column{0.5\textwidth}
gravity.tex:
\lstinputlisting[language=mytex, frame=single, firstline=6, lastline=10, firstnumber=6]{examples/gravity.tex}
refs.bib:
\lstinputlisting[language=mytex, frame=single, firstline=1, lastline=7, firstnumber=1]{examples/refs.bib}
\column{0.05\textwidth}
\vspace{3cm}
\arrowright
\column{0.45\textwidth}
\fbox{\includegraphics[width=0.9\textwidth]{examples/gravity.pdf}}
\vspace{0.5cm}
\end{columns}
\end{frame}

%%%%%%%%%%%%%%%%%%%%%%%%%%%%%%%%%%%%%%%%%%%%%%%%%%%%%%%%%%%%%%%%%%%%%%%%%%%%%%%%
\section{Beamer}
%%%%%%%%%%%%%%%%%%%%%%%%%%%%%%%%%%%%%%%%%%%%%%%%%%%%%%%%%%%%%%%%%%%%%%%%%%%%%%%%
\begin{frame}[fragile]
\frametitle{Beamer Hello World!}

\begin{columns}[T]
\column{0.5\textwidth}
beamer\_hello.tex:
\lstinputlisting[language=myTeX, frame=single]{examples/beamer_hello.tex}
\column{0.05\textwidth}
\vspace{3cm}
\arrowright
\column{0.45\textwidth}
\fbox{\includegraphics[width=0.9\textwidth]{examples/beamer_hello.pdf}}
\vspace{0.5cm}
\end{columns}
\end{frame}
%%%%%%%%%%%%%%%%%%%%%%%%%%%%%%%%%%%%%%%%%%%%%%%%%%%%%%%%%%%%%%%%%%%%%%%%%%%%%%%%
\begin{frame}[fragile]
\frametitle{Columns}

\begin{columns}[T]
\column{0.5\textwidth}
columns.tex:
\lstinputlisting[language=myTeX, frame=single, firstline=4, lastline=14, firstnumber=4]{examples/columns.tex}
\column{0.05\textwidth}
\vspace{3cm}
\arrowright
\column{0.45\textwidth}
\fbox{\includegraphics[width=0.9\textwidth]{examples/columns.pdf}}
\vspace{0.5cm}
\end{columns}
\end{frame}
%%%%%%%%%%%%%%%%%%%%%%%%%%%%%%%%%%%%%%%%%%%%%%%%%%%%%%%%%%%%%%%%%%%%%%%%%%%%%%%%
\begin{frame}[fragile]
\frametitle{Blocks}

\begin{columns}[T]
\column{0.5\textwidth}
blocks.tex:
\lstinputlisting[language=myTeX, frame=single, firstline=5, lastline=21, firstnumber=5]{examples/blocks.tex}
\column{0.05\textwidth}
\vspace{3cm}
\arrowright
\column{0.45\textwidth}
\fbox{\includegraphics[width=0.9\textwidth]{examples/blocks.pdf}}
\vspace{0.5cm}
\end{columns}
\end{frame}
%%%%%%%%%%%%%%%%%%%%%%%%%%%%%%%%%%%%%%%%%%%%%%%%%%%%%%%%%%%%%%%%%%%%%%%%%%%%%%%%



%%%%%%%%%%%%%%%%%%%%%%%%%%%%%%%%%%%%%%%%%%%%%%%%%%%%%%%%%%%%%%%%%%%%%%%%%%%%%%%%
%%%%%%%%%%%%%%%%%%%%%%%%%%%%%%%%%%%%%%%%%%%%%%%%%%%%%%%%%%%%%%%%%%%%%%%%%%%%%%%%
%%%%%%%%%%%%%%%%%%%%%%%%%%%%%%%%%%%%%%%%%%%%%%%%%%%%%%%%%%%%%%%%%%%%%%%%%%%%%%%%
\end{document}
%%%%%%%%%%%%%%%%%%%%%%%%%%%%%%%%%%%%%%%%%%%%%%%%%%%%%%%%%%%%%%%%%%%%%%%%%%%%%%%%
%%%%%%%%%%%%%%%%%%%%%%%%%%%%%%%%%%%%%%%%%%%%%%%%%%%%%%%%%%%%%%%%%%%%%%%%%%%%%%%%
